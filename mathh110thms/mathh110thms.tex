\documentclass[12pt]{article}
% \usepackage[left=2cm, right=2cm, top=1.5cm, bottom=1.5cm]{geometry}
\usepackage{amsmath}
\usepackage{amsthm}
\usepackage{amsfonts}
\usepackage{amssymb}
\usepackage{authblk}
\usepackage{tkz-euclide}
\usepackage{tikz}
\usepackage{changepage}
\usepackage{lipsum}
\usepackage{tree-dvips}
\usepackage{qtree}
\usepackage[linguistics]{forest}
\usepackage[hidelinks]{hyperref}
\usepackage{mathtools}
\usepackage{blindtext}
% \usepackage[cal=esstix,frak=euler,scr=boondox,bb= pazo]{mathalfa}
% the following 2 packages are used for changing the font. 
\usepackage{mathptmx}
\usepackage{mathrsfs}
\usepackage{graphicx}
\usepackage{setspace}
\graphicspath{{./images/}}
\allowdisplaybreaks
\allowbreak
\theoremstyle{definition}
\newtheorem{definition}{Definition}
\newtheoremstyle{named}{}{}{\itshape}{}{\bfseries}{.}{.5em}{\thmnote{#3's }#1}
\theoremstyle{named}
\newtheorem*{namedconjecture}{Distinct Factorizations Conjecture}
\newtheorem{conjecture}{Conjecture}
\DeclareMathOperator{\sech}{sech}
\DeclareMathOperator{\arcsec}{arcsec}
\DeclareMathOperator{\lcm}{lcm}
\DeclareMathOperator{\curl}{curl}
\DeclareMathOperator{\Res}{Res}
\DeclareMathOperator{\Aut}{Aut}
\DeclareMathOperator{\id}{id}
\DeclareMathOperator{\nul}{nul}
\newcounter{customDef}
\renewcommand{\thecustomDef}{\arabic{customDef}}
\newcommand{\Mod}[1]{\ (\mathrm{mod}\ #1)}
\begin{document}

\begin{center}
    Math H110 Theorems. 
\end{center}

\begin{enumerate}
    \item \textbf{Lemma. } Let $F$ be a field, $\lambda \in F$, $V$ a vector space over $F$ (denoted by $V/F$), $v \in V$. Then, if $\lambda v = 0$, then $\lambda=0$ or $v=0$. 
    \item \textbf{Lemma. } A vector space over a field is a module over a field. 
    \item \textbf{Theorem. } The intersection of a family of subspaces of a vector space $V$ is a subspace of $V$. 
    \item \textbf{Lemma. } Let $S=\{v_1,\dots,v_t\}$. Then the subspace of all linear combinations of the elements of $S$ is the $\mathrm{span}S$. 
    \item \textbf{Theorem. } Let $L=v_1,\dots,v_n$ be a list of vectors in a vector space $V$ over a field $F$ and let $T: F^n: \to V$ be linear transformation with $(\lambda_1,\dots,\lambda_n) \mapsto \lambda_1v_1 + \dots + \lambda_nv_n$. Then, we have the following: 
    \begin{enumerate}
        \item $L$ spans $V$ iff $T$ is onto. 
        \item $L$ is linearly independent iff $T$ is 1-1 iff $\nul T = \{0\}$. 
        \item $L$ is a basis iff $T$ is 1-1 and onto. 
    \end{enumerate}
    \item \textbf{Prop. } Consider $T: F^n \to V$ with $(\lambda_1,\dots,\lambda_n) \mapsto \lambda_1v_1 + \dots + \lambda_nv_n$, so $T(e_i)=v_i$ for all $i$. Then, $T$ is the unique linear map $F_n \to V$ that sends $e_i \mapsto v_i$ for all $i$. 
    \item \textbf{Theorem. } Every subspace $X$ of $V$ has complement. 
	\item \textbf{Lemma. } If $v_1,\dots,v_t$ is linearly dependent list, then there is an index $k$ such that $v_k \in \textrm{span}(v_1,\dots,v_{k-1},v_{k+1},\dots,v_t)$. Furthermore, the span of the list of length $t-1$ gotten by removing $v_k$ from the list is the same as the span of the original list. 
	\item \textbf{Prop. } In a finite-dimensional vector space, the length is of every linearly independent list of vectors is less than or equal to the length of every spanning list of vectors. 
	\item \textbf{Cor. } Two bases of $V$ have the same number of elements. 
	\item \textbf{Prop. } $X+Y$ is direct iff the null space of the sum map is $\{0\}$. 
	\item \textbf{Theorem. } Every subspace of a finite-dimensional vector space is finite-dimensional. 
	\item \textbf{Prop. } Every spanning list for a vector space can be pruned down to a basis of the space. 
	\item \textbf{Cor. } Every finite-dimensional vector space has a basis. 
	\item \textbf{Prop. } In a finite-dimensional vector space, every linearly independent list can be extended to a basis of the space. 
	\item \textbf{Major Theorem. } Every subspace of a finite-dimensional vector space has a complement. 
	\item \textbf{Prop. } Let $X,Y$ be subspaces of a finite-dimensional vector space $V$. Then: 
	\begin{enumerate}
		\item $\dim X + \dim Y = \dim V$. 
		\item $X \cap Y = \{0\}$. 
	\end{enumerate}
	Then, $V=X \oplus Y$. 
	\item \textbf{Prop. } $\dim(X \oplus Y) = \dim X + \dim Y$. 
	\item \textbf{Prop. } If $V$ is a finite-dimensional vector space (with $\dim V = n$), then every subspace has dimension at most $n$. 
	\item \textbf{Prop. } Let $\dim V = n$. Then, a linearly independent list of vectors of $V$ with length $n$ is a basis for $V$. 
	\item \textbf{Prop. } Let $\dim V = n$. Then, every spanning list for $V$ of length $n$ is a basis for $V$.
	\item \textbf{Lemma. } The list $(x_1,0),\dots,(x_t,0);(0,y_1),\dots,(0,y_k)$ of length $t+k$ is a basis of $X \times Y$. 
	\item \textbf{Cor. } $\dim (X \times Y) = \dim X + \dim Y$. 
	\item \textbf{Cor. } Let $T: V \to W$ be a linear map with $\dim V = d$. Then, $\textrm{rank}T \leq d$. 
	\item \textbf{Rank-Nullity Theorem. } $\dim V = \textrm{rank}V + \textrm{nullity}V$. 
	\item \textbf{Prop. } If $T: V \to W$ is 1-1, then $\textrm{nullity}T = 0$. 
	\item \textbf{Cor. } If $T: V \to W$ is 1-1 and onto, then $\dim V = \dim W$. 
	\item \textbf{Theorem. } The set of linear maps $V \to W$ is a vector space $L \cdot (F^n,W) \to T \longrightarrow (Te_1,\dots,Te_n) \in W^n$. 
	\item \textbf{Theorem. } $\dim (X+Y) = \dim X + \dim Y - \dim (X \cap Y)$. 
	\item \textbf{Cor. } $\dim (V/X) = \dim V - \dim X$. 
	\item \textbf{Theorem. } If $A$ is a rectangular matrix with elements in a field $F$, then row rank $A$ = column rank $A$. 
	\item \textbf{Prop. } Let $T: V \to W$ be 1-1. Then, $\dim W \geq \dim V$. 
	\item \textbf{Prop. } Let $T: V \to W$ be onto. Then, $\dim V \geq \dim W$. 
	\item \textbf{Prop. } Let $T: V \to W$ and $\dim V = \dim W$. Then, $T$ 1-1 iff $T$ onto iff $T$ bijective iff $T$ invertible. 
	\begin{center}
		\hrule
	\end{center}
	\item \textbf{Lemma. } Let $V$ be a finite-dimensional vector space and $U$ a subspace of $V$. Then, $\dim U_0 = \dim V - \dim U$. 
	\item \textbf{Theorem. } Every linear functional on a subspace of $V$ can be extended to $V$. 
	\item \textbf{Note. } Annihilator is the dual of the quotient subspace. 
	\item \textbf{Theorem. } Let $T: V \to W$ and $T': W' \to V'$. THen $\mathscr{M}(T)$ and $\mathscr{M}(T')$ are transposes of each other. 
	\item \textbf{Lemma. } $U^0$ has dimension $\dim V - \dim U$. 
	\item \textbf{Cor. } The annihilator of $U$ is $\{0\}$ iff $U = V$. The annihilator of $U$ is $V$ iff $U = \{0\}$. 
	\item \textbf{Prop. } If $T: V \to W$ is a linear map, then the null space of $T'$ is the annihilator of the range of $T$. We have $\textrm{ann}(\textrm{range}T) = \{\psi: W \to F \mid \phi(Tv)=0 \textrm{ for all } v \in V, T'(\psi)(v)=0, T'\psi = 0, \phi \in \nul(T')\}$. 
	\item \textbf{Cor. } If $T: V \to W$ is a linear map between finite-dimensional $F$-vector spaces, then $\dim \nul(T') = \dim \nul(T) + \dim W - \dim V$. 
	\item \textbf{Cor. } The linear map $T$ is onto iff $T'$ is 1-1. 
	\item \textbf{Cor. } If $T: V \to W$ is a linear map between finite-dimensional vector spaces, then $T'$ and $T$ have equal ranks. 
	\item \textbf{Cor. } We have $\textrm{range}T = (\nul T)^0$. 
	\item \textbf{Theorem. } Let $F$ be a finite field with $q = |F|$. Then, $a^q=a$ for all $a \in F$. 
	\item \textbf{Theorem. } If $F$ is a finite field, then $|F|=p^n$ for some prime $p$ and integer $n \geq 1$. 
\end{enumerate}

\end{document}
