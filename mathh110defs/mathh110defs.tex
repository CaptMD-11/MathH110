\documentclass[12pt]{article}
% \usepackage[left=2cm, right=2cm, top=1.5cm, bottom=1.5cm]{geometry}
\usepackage{amsmath}
\usepackage{amsthm}
\usepackage{amsfonts}
\usepackage{amssymb}
\usepackage{authblk}
\usepackage{tkz-euclide}
\usepackage{tikz}
\usepackage{changepage}
\usepackage{lipsum}
\usepackage{tree-dvips}
\usepackage{qtree}
\usepackage[linguistics]{forest}
\usepackage[hidelinks]{hyperref}
\usepackage{mathtools}
\usepackage{blindtext}
% \usepackage[cal=esstix,frak=euler,scr=boondox,bb= pazo]{mathalfa}
% the following 2 packages are used for changing the font. 
\usepackage{mathptmx}
\usepackage{mathrsfs}
\usepackage{graphicx}
\usepackage{setspace}
\graphicspath{{./images/}}
\allowdisplaybreaks
\allowbreak
\theoremstyle{definition}
\newtheorem{definition}{Definition}
\newtheoremstyle{named}{}{}{\itshape}{}{\bfseries}{.}{.5em}{\thmnote{#3's }#1}
\theoremstyle{named}
\newtheorem*{namedconjecture}{Distinct Factorizations Conjecture}
\newtheorem{conjecture}{Conjecture}
\DeclareMathOperator{\sech}{sech}
\DeclareMathOperator{\arcsec}{arcsec}
\DeclareMathOperator{\lcm}{lcm}
\DeclareMathOperator{\curl}{curl}
\DeclareMathOperator{\Res}{Res}
\DeclareMathOperator{\Aut}{Aut}
\DeclareMathOperator{\id}{id}
\DeclareMathOperator{\nul}{nul}
\DeclareMathOperator{\End}{End}
\newcounter{customDef}
\renewcommand{\thecustomDef}{\arabic{customDef}}
\newcommand{\Mod}[1]{\ (\mathrm{mod}\ #1)}
\begin{document}

\begin{center}
    Math H110 Definitions. 
\end{center}

\begin{enumerate}
    \item \textbf{Endomorphism. } An endomorphism is a group homomorphism from a set to itself (NOTE: does not have to be invertible.)
    \item \textbf{End V. } The symbol $\End V$ is the set of all endomorphisms on $V$ (and multiplication on $\End V$ is defined to be function composition). 
    \item \textbf{F-Module. } An $F-$module is a generalization of vector spaces over rings. 
    \item \textbf{Subspace. } Let $V$ be a vector space. $X$ is a subspace of $V$ if $X \subseteq V$ and closed under all relevant operations of $V$, $X \neq \emptyset$, and $X \ni 0$. 
    \item \textbf{Linear Map / Linear Transformation. } Let $V$ be a vector space over a field $F$ with $v,w \in V$. Let $T$ be a map on $V$ with $T(v+w) = T(v) + T(w)$ and $T(\lambda v) = \lambda T(v)$ for all $\lambda \in F$. Then, $T$ is called a linear map or linear transformation. 
    \item \textbf{Linear Operator. } If $T$ is linear transformation on a vector spaces $V$ with $T: V \to V$, then $T$ is linear operator on $V$. 
\end{enumerate}

\end{document}