\input{~/normal-preamble.tex}

\begin{itemize}
	\item 1C. 
	\item 5 | Is $R^2$ a subspace of the complex vector space $C^2$? | No, check scalar mutliplication by $i$. 
	\item 7 | prove/disprove: If $U$ is a nonempty subset of $R^2$ that is closed under addition and taking additive inverses, then $U$ is a subspace. | No, take $U = \{(x,0) \mid x \in Q\}$. 
	\item 8 | want an example of a subset $U$ of $R^2$ that satisfies scalar mutliplication but not vector addition | take $x$-axis, $y$-axis. 
	\item 13 | prove that union of three subspaces of $V$ is a subspace iff one contains the other two | suffices to show if $W$ is the union of three of its subspaces, then then one of the three subspaces is contained in the union of the other two, by applying result of prob 1c.12. 
	\item 18 | does the operation of addition on subspaces of $V$ have additive identity? which subspaces have inverses? | add. id. is $\{0\}$ and only $\{0\}$ has additive inverse. 
	\item 19 | prove/disprove: If $V_1,V_2,U$ subspaces of $V$, then $V_1 + U = V_2 + U$ implies $V_1 = V_2$ | false. take $V_1$,$V_2$,$U$ to be $x$-axis, $y$-axis, $y=x$ respectively. 
	\item 23 | prove/disprove: If $V_1,V_2,U$ subspaces of $V$ so that $V=V_1 \oplus U = V_2 \oplus U$, then $V_1=V_2$. | use same counterexample as in problem above. 
	\item 2A. 
	\item 7b | Show if $C$ is a vector space over $C$, then $1+i,1-i$ is L.D. | for L.D. relation, choose coeffs $a=i$ and $b=1$ for $1+i,1-i$, resp. 
	\item 17 | show that $V$ is inf-dim iff there is a sequence $v_1,v_2,\dots \in V$ so that $v_1,\dots,v_m$ l.i. for all positive integers $m$. | for reverse direction, prove contrapositive. 
	\item 19 | show that real vector space of all cts real values functions on $[0,1]$ is inf-dim. | look at polynomials. 
	\item 5 | let $V$ be finite-dim and $V=U+W$. show there is a basis of $V$ consisting of vectors from $U \cup W$. | concatenate bases of $U$ and $W$.
	\item 2C. 
	\item 8 | let $v_1,\dots,v_m$ be linearly independent in $V$ and $w \in V$. show $\dim span(v_1+w,\dots,v_m+w) \geq m-1$. | look at $v_1-v_2,\dots,v_1-v_m$ and it is contained in the dimspan. 
	\item 10 | let $m$ be a positive integer. for $0 \leq k \leq m$, let $p_k = x^k(1-x)^{m-k}$. show $p_0,\dots,p_m$ is a basis of $P_m(F)$. | set linear dependence relation and choose $x$-values to substitute. 
	\item 11 | Let $U$,$W$ be 4-dim subspaces of $C^6$. show there exist two vectors in $U \cap W$ that aren't scalar multiples of each other. | equivalent to showing $\dim (U \cap W) \geq 2$. 
	\item 16 | let $V$ be finite-dim and $U$ a strict subspace of $V$. let $n=\dim U$ and $m=\dim V$. show there are $n-m$ subsapces of $V$ each with dim 1 whose intersectin is $U$ | fix a basis of $U$ and extend it to basis of $V$ and construct subspaces that each 'delete' a vector in the extension of $U$. 
	\item 3A. 
	\item 8 | give a function $\phi: R^2 \to R$ so that $\phi(\lambda v) = \lambda \phi(v)$ but $\phi$ not linear | take $\phi(x,y) = (x^3 + y^3)^{1/3}$. 
	\item 10 | prove/disprove: if $q \in P(R)$ and $T: P(R) \to P(R)$ by $Tp = q \circ p$, then $T$ is linear | false. take $q =x^2$. 
	\item 11 | let $V$ be finitedim and $T \in L(V)$. show $T = \lambda I$ iff $ST=TS$ for all $S \in L(V)$. | for reverse direction, try contrapositive and look at $\ker T$. 
	\item 12 | let $U$ be a strict subspace of $V$. Let $S \in L(U,W)$ and $S \neq 0$. Let $T(v) = Sv$ if $v \in U$ and $Tv=0$ if $v \notin U$. show $T$ is not linear. | pick $v_1 \in U$ and $v_2 \notin U$. 
	\item 17 | let $V$ be finitedim. show the only two-sided ideals of $L(V)$ are $\{0\}$ and $L(V)$. | let $w$ be so that $Tw \neq 0$. let $S_k: V \to V$ that sends $v_j$ to 0 for $j \neq k$ and $v_k$ to $w$. put $R_k$ so that $R_k(Tw) = v_k$, and look at $R_kTS_kv_j$. 
	\item 3B. 
	\item 15 | Suppose there is a linear map on $V$ so that both null space and range of it are finitedim. show that $V$ is finite dim. | look at basis $Tv_1,\dots,Tv_n$ for range and $w_1,\dots,w_k$ for null space. 
	\item 19 | Let $W$ be finitedim and $T \in L(V,W)$. show $T$ is 1-1 iff there exists $S \in L(W,V)$ so that $ST=I$ on $V$. | letting $T: V \to W$ be 1-1 and looking at $U=\range T$, put $S: U \to V$ as the inverse of $T$ and extend to $S: W \to V$. 
	\item 20 | let $W$ be finite-dim and $T \in L(V,W)$. Show $T$ onto iff there exists $S \in L(W,V)$ so that $TS=I$ on $W$. | use ontoness of $T$ and look at restriction of $T$ to $X$, the complement of $null T$. do isomorphism $X \cong W$ and put $S: W \to X$ so that $TS=I$. 
	\item 3C. 
	\item 5 | Let $V,W$ be finitedim and $T \in L(V,W)$. show there is a basis of $V$ and a basis of $W$ so that in these bases, all entries of $M(T)$ are 0 except those in entries row $k$ col $k$ if $1 \leq k \leq \range T$. | $U=\nul T$ and $X$ is complement to $U$ in $V$. put bases of $X$ and $U$. find bases of $\range T$ and complete to get basis of $W$. 
	\item 6 | Let $v_1,\dots,v_n$ be basis of $V$ and $W$ is finitedim and let $T \in L(V)$. show there is a basis $w_1,\dots,w_m$ so that all entries of $M(T)$, in these bases, are 0 except possibly a 1 in the first row, first col. | first column is $Tv_1$. consider when $Tv_1 = 0, \neq 0$. put basis $W = span(Tv_1,w_2,\dots,w_m)$. 
	\item 7 | Let $w_1,\dots,w_n$ a basis of $W$ and $V$ finitedim and $T \in L(V,W)$. Show there is a basis $v_1,\dots,v_m$ of $V$ so that all entries in first row of $M(T)$, in these bases, are 0 except possilby a 1 in first row, first col. | Look at $T': W' \to V'$ and apply 3c.6 result. 
	\item 3D. 
	\item 10 | Let $V,W$ be finite dim and $U \subseteq V$. put $E = \{T \in L(V,W) \mid U \subseteq \nul T\}$. find a formula for $\dim E$ in terms of $\dim V, \dim U, \dim W$. | put $\Phi: L(V,W) \to L(U,W)$ by $\phi(T) = T \mid_U$ and find range and null space. 
	\item 19 | let $V$ be finitedim and $T \in L(V)$. show $T$ has same matrix with respect to every basis of $V$ iff $T = \lambda I$. | fix a matrix of $T$ and for basis $v_1,\dots,v_m$ of $V$, $v_1,\dots,(1/2)v_k,\dots,v_m$ is also basis; scale and edit. 
	\item 20 | let $q \in P(R)$. show there is a polynomial $p \in P(R)$ so that $q(x) = (x^2 + x)p''(x) + 2xp'(x) + p(3)$. | define $T: P(R) \to P(R)$ by $Tp = (x^2 + x)p''(x) + 2xp'(x) + p(3)$ and show $T$ is 1-1 (and by finitedim of domain,codomain) thus $T$ is onto. 
	\item 3E. 
	\item 9 | Show a nonempty subset $A$ of $V$ is a translate of some subspace of $V$ iff $\lambda v + (1-\lambda)w \in A$ for all $v,w \in A$, $\lambda \in F$. | for converse, fix $x \in A$ attempt for $A = x + U$, where $U = \{a-x \mid a \in A\}$. 
	\item 17 | Let $U$ be a subspace of $V$ so that $\dim V/U = 1$. show there is $\phi \in L(V,F)$ so that $\nul\phi = U$. | put $T: V/U \to F$ that sends everything to $1 \in F$. let $\phi$ be composite of $\pi: V \to V/U$ and $T$. 
	\item 3F. 
	\item 6 | let $\phi,\beta \in V'$. show $\nul\phi \subseteq \nul\beta$ iff there is $c \in F$ so that $\beta = c\phi$. | by a previous problem, there is $S \in L(F)$ so that $\beta = S\phi$. 
	\item 26 | let $V$ be finitedim and $\Omega$ be a subspace of $V'$. show $\Omega = \{v \in V \mid \phi(v)=0 \forall \phi \in \Omega\}^0 = U^0$. | show $U = \cap_{i=1}^{m} (\nul \phi_i)$. 
	\item 5A. 
	\item 15 | Let $V$ be finitedim, $T \in L(V)$, and $\lambda \in F$. show $\lambda$ is an eigenvalue of $T$ iff $\lambda$ is an eigenvalue of $T'$ | use chain of if and only ifs. 
	\item 20 | let $S \in L(F^\infty)$ be backwards shift by $S(z_1,z_2,\dots) = (z_2,z_3,\dots)$. show each $f \in F$ is an eigenvalue and find all eigenvectors. | look at $(1,\lambda,\lambda^2,\dots)$. 
	\item 28 | let $V$ be finitedim and $T \in L(V)$. show $T$ has at most $1 + \dim\range T$ distinct eigenvalues. | put distinct eigenvalues/vectors and for nonzero eigenvalues, look at $v_i = T((1\lambda_i)v_i)$ and linear independence and range. 
	\item 39 | Let $V$ be finitedim and $T \in L(V)$. show $T$ has eigenvalue iff there is a subspace of $V$ of $\dim V - 1$ that is $T$-invariant. | one direction: use fact eigenvalues of $T_{V/U}$ are eigenvalues of $T$. other direction: if $\lambda$ eigenvalue, then $T-\lambda I$ noninvertible so its range has dim $< \dim V$. if $X = \range T$, every subspace $W$ of $V$ with $X \subseteq W \subseteq V$ is $T$-invariant. 
	\item 5B. 
	\item 2 | let $V$ be a complex vector space and $T \in L(V)$ have no eigenvalues. show every subspcae of $V$ invariant under $T$ is $\{0\}$ or infinite-dim. | Take instead a finitedim $X \subseteq V$; it has an eigenvector. 
	\item 3 | let $n \in \mathbb{Z}_{>0}$ and $T \in L(F^n)$ by $T(x_1,\dots,x_n) = (x_1+\dots+x_n,\dots,x_1+\dots+x_n)$. find all eigenvalues/vectors and minimal polynomial of $T$ | $\range T = \{(a,\dots,a)\}$. 
	\item 4 | let $F=C$, $T \in L(V)$, $p \in P(C)$ is a nonconstant polynomial and $\alpha \in C$. show $\alpha$ is eigenvalue of $p(T)$ iff $\alpha = p(\lambda)$ for some eigenvalue $\lambda$ of $T$ | one direction: $p(T)v = p(\lambda)v$. other direction: $T$ is upper-triangular in some basis of $V$; look at diagonal and look at $p(T)$. 
	\item 5 | for above question, find an example where instead $V = R^2$. | $\begin{pmatrix} 0 & -1 \\ 1 & 0 \end{pmatrix}$. 
	\item 7 | show if $V$ finitedim and $S,T \in L(V)$, then if at least one of $S,T$ inveritible, then minimal poly of $ST$ equals that of $TS$. | first show $Sp(T)S^{-1} = p(STS^{-1})$. then $T,STS^{-1}$ have same minimal poly. replace $T$ by $TS$. 
	\item 10 | let $V$ be finitedim and $T \in L(V)$. show $span(v,Tv,\dots,T^m v)=span(v,Tv,\dots,T^{\dim V - 1}v)$ for all $m \geq \dim V - 1$. | note $v,Tv,\dots,T^{m}v$ has dim $m$. 
	\item 19 | let $V$ be finitedim and $T \in L(V)$. let $\epsilon = \{q(T) \mid q \in P(F)\}$. show $\dim \epsilon$ = degree of minimal poly of $T$ | observe $F[x] / (\nul \alpha)$, algebra. 
	\item 25 | $V$ finitedim, $T \in L(V)$, $U \subseteq V$ invariant under $T$. show minimal poly of $T$ is poly multiple of minimal poly of $T_{V/U}$. also show (min poly of $T \mid_U$) x (min poly of $T_{V/U}$) is poly multiple of min poly of $T$. | first part: if $m$ is min poly of $T$, then $m(T \mid_U)$ is poly multiple of min poly of $T \mid_U$, similary for $T_{V/U}$. second part: let $g$ be min poly of $T_{V/U}$ and $f$ be min poly of $T \mid_U$ and show $(fg)(T)=0$. $g(T)$ is 0 map on $V/U$ and $f(T)$ maps $U$ to $\{0\}$. 
	\item 5C. 
	\item 7 | $V$ finitedim, $T \in L(V)$, and $v \in V$. show there is unique monic poly $p_v$ of smallest degree so that $p_v(T)v=0$. also show min poly of $T$ is a poly mult of $p_v$. | first part: $I = \{f(x) \mid f(T)v=0\}$, it contains 0 and closed under addition, and 'external multiplication' and use well-ordering. 
	\item 5D. 
	\item 2 | let $T \in L(V)$ have diagonal matrix $A$ corresponding to some basis of $V$. show that if $\lambda \in F$, then $\lambda$ appears on diag of $A$ exactly $\dim E(\lambda,T)$ times. | $E(\lambda,T) = \nul (T-\lambda I)$ and look at matrix multiplication. 
	\item 3 | $V$ finitedim, $T \in L(V)$ diagonalizable. show $V = \nul T \oplus \range T$. | look at eigenvalues that are 0 and eigenvalues that are nonzero. 
	\item 5 | $V$ finitedim complex vector space, $T \in L(V)$ and $V = \nul (T - \lambda I) \oplus \range (T - \lambda I)$ for all $\lambda \in C$. show $T$ diagonalizable. | do induction on $\dim V$. 
	\item 19 | prove/disprove: if $T \in L(V)$ and $U \subseteq V$ is invariant under $T$ so that $T \mid_U$ and $T_{V/U}$ are diagonalizable, then $T$ diagonalizable. | false: take $\begin{pmatrix} 0 & 1 \\ 0 & 0 \end{pmatrix}$. 
	\item 5E. 
	\item 2 | let $\epsilon$ be subset of $V$ where every $T \in \epsilon$ is diagonalizable. show there is a basis of $V$ with respect to which every $T \in \epsilon$ has diag matrix iff every pair $S,T \in \epsilon$ commutes. | converse: look at direct sum of operators, eigenspaces, and restrictions. 
	\item 6 | $V$ finitedim nonzero complex vector space and $ST=TS$. show there exist $\alpha,\lambda \in C$ so that $\range (S-\alpha I) + \range (T-\lambda I) \neq V$. | look at 2 upper triangular matrices, one with $\alpha$ in bottom left corner and another with $\lambda$ in bottom left corner. 
	\item 10 | want commuting operators $S,T$ so that $S+T$ has an eigenvalue that is not sum of eigenvalue of $S$ and eigenvalue of $T$, and similarly for $ST$. | let $S = \begin{pmatrix} 0 & -1 \\ 1 & 0 \end{pmatrix}, T = -S$. 
	\item 6A. 
	\item 
\end{itemize}

\end{document}
